%
% File acl2017.tex
%
%% Based on the style files for ACL-2015, with some improvements
%%  taken from the NAACL-2016 style
%% Based on the style files for ACL-2014, which were, in turn,
%% based on ACL-2013, ACL-2012, ACL-2011, ACL-2010, ACL-IJCNLP-2009,
%% EACL-2009, IJCNLP-2008...
%% Based on the style files for EACL 2006 by 
%%e.agirre@ehu.es or Sergi.Balari@uab.es
%% and that of ACL 08 by Joakim Nivre and Noah Smith

\documentclass[11pt,a4paper]{article}
\usepackage[hyperref]{acl2017}
\usepackage{times}
\usepackage{latexsym}

\usepackage{url}

%\aclfinalcopy % Uncomment this line for the final submission
%\def\aclpaperid{***} %  Enter the acl Paper ID here

%\setlength\titlebox{5cm}
% You can expand the titlebox if you need extra space
% to show all the authors. Please do not make the titlebox
% smaller than 5cm (the original size); we will check this
% in the camera-ready version and ask you to change it back.

\newcommand\BibTeX{B{\sc ib}\TeX}

\title{Performative Commands in Collaborative Dialogue}

\author{First Author \\
  Affiliation / Address line 1 \\
  Affiliation / Address line 2 \\
  Affiliation / Address line 3 \\
  {\tt email@domain} \\\And
  Second Author \\
  Affiliation / Address line 1 \\
  Affiliation / Address line 2 \\
  Affiliation / Address line 3 \\
  {\tt email@domain} \\}

\date{}

\begin{document}
\maketitle

\section{Introduction}

There are many ways to use language to compel others to action, perhaps the most straightforward of which is to use the imperative mood--- e.g. ``Please come here". The speech act of commanding can also be performed with a broad set of non-imperative utterance types. For example, the English utterances 
``You need to come here," ``If you could come here that would be great," ``I think you should come here," and ``Come over here?" can all be used to elicit action of an addressee. While subtly distinct, each of these in some way conveys a positive normative attitude of the speaker towards the proposition \textit{come(you)}, a fact which could be straightforwardly leveraged to perform dialogue act classification.

This is as opposed to a more elusive subclass of performative commands: Locative utterances like ``The dog is outside," whose semantics makes no such reference to an agent or a specific action, but which can be interpreted as commands in certain discourse contexts. 

Consider for example a hypothetical in which two people are setting up a room for a conference and must find chairs elsewhere in the building. One walks into the room carrying two chairs and, before putting them down says, ``There is a chair in the room next door." In one possible scenario, the addressee realizes he has the capacity to act on this information and go fetch the chair in question. In another, he simply stands where he is, and in response the speaker puts down the chairs she is carrying, and exasperatedly fetches the chair she had previously mentioned.

Despite that the utterance made no reference to the addressee as the agent of some action with respect to the chair, neither the case where he infers that the speaker wants him to act, or the case where the speaker gets exasperated that he failed to do so, would be unreasonable in the course of a natural, cooperative dialogue. This begs the question: In the absence of any explicit agent or action, how does something like a locative construction with an expletive subject function like a command? 

We explore this question using the Cards Corpus, a corpus of transcripts from a game in which pairs of players must collaboratively collect cards scattered on a gameboard. The Cards Corpus- is optimal for questions of this sort, because it records both the actions taken and the utterances made during the course of a game. Where commands are concerned, we can observe who acts in response to an utterance, and test hypotheses about the discourse conditions surrounding an utterance-action exchange.

In a first study, we examine the distribution of three classes of commands---imperative, non-locative performative, and locative---surrounding a fixed set of common actions taken in the Cards game. We find that for most actions the imperative strategy 
is the most prominent, but that the locative command strategy predominates for a very specific subset of actions. For example, when a player says ``The two of hearts is located in the top right corner" and intends that her game partner act on that information by fetching the card in question.

Whether these locative constructions are to be interpreted as commands is ambiguous. In a second Cards game study, we demonstrate that this ambiguity is resolved by features of the \textit{common ground}: here, the state of a game as reflected by the utterances made by both players up to a specific point in time. We describe regressors of addressee follow-up action, and hence of the contextual interpretation of locatives as commands.

These results agree with other research demonstrating that natural language understanding in embodied, multi-agent settings is heavily context-dependent. 
It remains to be seen how such pragmatic reasoning might be integrated into a more complete model of embodied speech production.

\section{Related Work}




\end{document}
