%
% File acl2017.tex
%
%% Based on the style files for ACL-2015, with some improvements
%%  taken from the NAACL-2016 style
%% Based on the style files for ACL-2014, which were, in turn,
%% based on ACL-2013, ACL-2012, ACL-2011, ACL-2010, ACL-IJCNLP-2009,
%% EACL-2009, IJCNLP-2008...
%% Based on the style files for EACL 2006 by 
%%e.agirre@ehu.es or Sergi.Balari@uab.es
%% and that of ACL 08 by Joakim Nivre and Noah Smith

\documentclass[11pt,a4paper]{article}
\usepackage[hyperref]{acl2017}
\usepackage{times}
\usepackage{latexsym}

\usepackage{url}

%\aclfinalcopy % Uncomment this line for the final submission
%\def\aclpaperid{***} %  Enter the acl Paper ID here

%\setlength\titlebox{5cm}
% You can expand the titlebox if you need extra space
% to show all the authors. Please do not make the titlebox
% smaller than 5cm (the original size); we will check this
% in the camera-ready version and ask you to change it back.

\newcommand\BibTeX{B{\sc ib}\TeX}

\title{Performative Commands: A Corpus Study}

\author{First Author \\
  Affiliation / Address line 1 \\
  Affiliation / Address line 2 \\
  Affiliation / Address line 3 \\
  {\tt email@domain} \\\And
  Second Author \\
  Affiliation / Address line 1 \\
  Affiliation / Address line 2 \\
  Affiliation / Address line 3 \\
  {\tt email@domain} \\}

\date{}

\begin{document}
\maketitle

\section{Introduction}

There are many ways to use language to compel others to action. Many languages have a dedicated tense---the imperative, e.g. ``come here"---for this purpose. The speech act of commanding can also be performed with a more general set of (non-imperative) utterance types. For example, the English utterances 
``You need to come here," ``If you could come here that would be great," ``I think you should come here," and ``Come over here?" can all be used to elicit action of an addressee. While subtly distinct, each of these in some way conveys a positive normative attitude of the speaker towards the proposition \textit{come(you)}, a fact which could be straighforwardly leveraged to perform dialogue act classification.

This is as opposed to a more elusive subclass of performative commands that has received less attention in the literature: Locative utterances like ``The dog is outside," whose semantics makes no such reference to an agent or a specific action, but which can be interpreted as commands in certain discourse contexts. 

Consider for example a hypothetical in which we two are setting up a room for a conference and we need to find chairs elsewhere in the building. I walk into the room carrying two chairs and, before putting them down say, ``There is a chair in the room next door." In one possible scenario, you realize you have the capacity to act on this information and go fetch the chair in question. In another, you simply stand where you are, and in response I put down the chairs I am carrying, and exasperatedly fetch the chair I had prevously mentioned.

Despite that my utterance made no mention of you as the agent of some action with respect to the chair, neither the case where you infer that I want you to act, or the case where I get exasperated that you failed to do so, would be unreasonable in the course of a natural, cooperative dialogue. This begs the question: In the absence of any explicit agent or action, how does something like a locative construction with an expletive subject function like a command? 


\end{document}
